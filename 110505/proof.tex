\documentclass[11pt]{article}

\pdfpagewidth 8.5in
\pdfpageheight 11in

\setlength\topmargin{0in}
\setlength\headheight{0in}
\setlength\headsep{0.2in}
\setlength\textheight{8in}
\setlength\textwidth{6in}
\setlength\oddsidemargin{0in}
\setlength\evensidemargin{0in}
\setlength\parindent{0.25in}
\setlength\parskip{0.1in} 
 
\usepackage{amssymb}
\usepackage{amsfonts}
\usepackage{amsmath}
\usepackage{mathtools}
\usepackage{amsthm}

      \theoremstyle{plain}
      \newtheorem{theorem}{Theorem}
      \newtheorem{lemma}[theorem]{Lemma}
      \newtheorem{corollary}[theorem]{Corollary}
      \newtheorem{proposition}[theorem]{Proposition}
      \newtheorem{conjecture}[theorem]{Conjecture}
      \newtheorem{question}[theorem]{Question}
      \newtheorem{example}[theorem]{Example}
      
      \theoremstyle{definition}
      \newtheorem{definition}[theorem]{Definition}
      
      \theoremstyle{remark}
      \newtheorem{remark}[theorem]{Remark}


% Strategy uparrow shortcuts
\newcommand{\win}{\uparrow}
\newcommand{\prewin}{\uparrow_{\text{pre}}}
\newcommand{\markwin}{\uparrow_{\text{mark}}}
\newcommand{\tactwin}{\uparrow_{\text{tact}}}
\newcommand{\ktactwin}[1]{\uparrow_{#1\text{-tact}}}
\newcommand{\kmarkwin}[1]{\uparrow_{#1\text{-mark}}}
\newcommand{\codewin}{\uparrow_{\text{code}}}
\newcommand{\limitwin}{\uparrow_{\text{limit}}}

\newcommand{\oneptcomp}[1]{#1^*}

\newcommand{\congame}[2]{Con_{O,P}(#1,#2)}
\newcommand{\clusgame}[2]{Clus_{O,P}(#1,#2)}

\newcommand{\lfkpgame}[1]{LF_{K,P}(#1)}
\newcommand{\lfklgame}[1]{LF_{K,L}(#1)}

\newcommand{\pfgame}[1]{PF_{F,C}(#1)}

\newcommand{\sigmaprodr}[1]{\Sigma\mathbb{R}^{#1}}
\newcommand{\sigmaprodtwo}[1]{\Sigma2^{#1}}

\newcommand{\<}{\langle}
\renewcommand{\>}{\rangle}

\newcommand{\rest}{\restriction}

\newcommand{\largecomment}[1]{}



\begin{document}

\begin{definition}
Take an integer $b\geq 2$, and let $G(x)$ be a two-player game defined for any real number $x>1$ as follows: during each Round of the game, Player I first chooses an integer $n$ with $2\leq n \leq b$. Player II closes the Round by doing the same. The first player to choose a number such that the product of all chosen numbers is $x$ or greater is declared the winner.
\end{definition}

\begin{theorem}
For numbers $x$ of the form
  \[
    2^kb^k < x  \leq 2^kb^{k+1}
  \]
Player I has a winning strategy in $G(x)$ which resolves during Round $k$.

For numbers $x$ of the form
  \[
    2^kb^{k+1} < x \leq 2^{k+1}b^{k+1}
  \]
Player II has a winning strategy in $G(x)$ which resolves during Round $k$.
\end{theorem}

\begin{proof}
We begin by inspecting the case where $k=0$. 

For numbers $x$ of the form
  \[
    2^0b^0 < x  \leq 2^0b^{0+1} \Leftrightarrow 1 < x \leq b
  \]
Player I chooses $b$ as her first and only move in Round $0$, winning $G(x)$.

For numbers $x$ of the form
  \[
    2^0b^{0+1} < x  \leq 2^{0+1}b^{0+1} \Leftrightarrow b < x \leq 2b
  \]
Player I chooses some $m$ with $2 \leq m \leq b$, and has not won the game. Player II finishes Round $0$ by
  responding with $b$, and as $x \leq 2b \leq mb$, Player II has won $G(x)$.

We now assume the theorem holds for $k$ by induction, and prove it holds for $k+1$.

We first wish to prove that Player I has a winning strategy for $G(x)$ resolving in Round $k+1$, where
  \[
    2^{k+1}b^{k+1} < x  \leq 2^{k+1}b^{k+2}
  \]

Let $m = \left\lceil\frac{x}{2^{k+1}b^{k+1}}\right\rceil$, where $\lceil \cdot \rceil$ denotes the ceiling function, which returns the least integer greater than or equal to the input. Since $x > 2^{k+1}b^{k+1}$, $m$ is at least $2$. Since $x \leq 2^{k+1}b^{k+2}$, $m$ is at most $b$. Thus $m$ is a legal move in $G(x)$.

We observe that we may express
  \[
    m = \frac{x+\epsilon}{2^{k+1}b^{k+1}}
  \]
where $\epsilon$ is some integer which satisfies $0 \leq \epsilon < 2^{k+1}b^{k+1}$.

Then, we see that
  \[
    \frac{x}{m} = 
    \frac{x}{\frac{x+\epsilon}{2^{k+1}b^{k+1}}} =
    \frac{x}{x+\epsilon}2^{k+1}b^{k+1}
  \]

Since $0 \leq \epsilon$:
  \[
    \frac{x}{m} =
    \frac{x}{x+\epsilon}2^{k+1}b^{k+1} \leq
    \frac{x}{x}2^{k+1}b^{k+1} =
    2^{k+1}b^{k+1}
  \]

Since $\epsilon < 2^{k+1}b^{k+1} < x$:
  \[
    \frac{x}{m} =
    \frac{x}{x+\epsilon}2^{k+1}b^{k+1} >
    \frac{x}{2x}2^{k+1}b^{k+1} =
    \frac{1}{2}2^{k+1}b^{k+1} =
    2^kb^{k+1}
  \]

As a result, $2^kb^{k+1} < \frac{x}{m} \leq 2^{k+1}b^{k+1}$, and by the induction hypothesis $G(\frac{x}{m})$ can be won by Player II during Round $k$. Let $s$ be the function representing this winning strategy. The game $G(\frac{x}{m})$ proceeds as follows:

\begin{center}\begin{tabular}{c|c|c}
$G(\frac{x}{m})$ & Player I & Player II \\\hline
Round $0$ & $n_0$ & $s(n_0)$ \\\hline
Round $1$ & $n_1$ & $s(n_0, n_1)$ \\\hline
$\vdots$ & $\vdots$ & $\vdots$ \\\hline
Round $k$ & $n_k$ & $s(n_0,\dots,n_k)$
\end{tabular}\end{center}

Since Player II won this game, we know 
  \[
    n_0s(n_0)\dots n_{k-1}s(n_0,\dots,n_{k-1})n_k < \frac{x}{m}
  \]
but 
  \[
    n_0s(n_0)\dots n_{k-1}s(n_0,\dots,n_{k-1})n_ks(n_0,\dots,n_k) \geq \frac{x}{m}
  \]

We claim the following strategy for Player I guarantees a victory in $G(x)$: Player I opens with $m$, and then uses Player II's strategy $s$ for the rest of the game.

\begin{center}\begin{tabular}{c|c|c}
$G(x)$ & Player I & Player II \\\hline
Round $0$ & $m$ & $n_0$ \\\hline
Round $1$ & $s(n_0)$ & $n_1$ \\\hline
$\vdots$ & $\vdots$ & $\vdots$ \\\hline
Round $k$ & $s(n_0,\dots,n_{k-1})$ & $n_k$ \\\hline
Round $k+1$ & $s(n_0,\dots,n_k)$ & -
\end{tabular}\end{center}

Player I has now won, as
  \[
    mn_0s(n_0)\dots n_{k-1}s(n_0,\dots,n_{k-1})n_k < x
  \]
and 
  \[
    mn_0s(n_0)\dots n_{k-1}s(n_0,\dots,n_{k-1})n_ks(n_0,\dots,n_k) \geq x
  \]

We conclude by showing that Player II has a winning strategy for $G(x)$ resolving in Round $k+1$, where
  \[
    2^{k+1}b^{k+2} < x  \leq 2^{k+2}b^{k+2}
  \]

Let $2 \leq n \leq b$. We see that
  \[
    \frac{x}{n} \leq
    \frac{x}{2} \leq
    \frac{2^{k+2}b^{k+2}}{2} =
    2^{k+1}b^{k+2}
  \]
and
  \[
    \frac{x}{n} \geq
    \frac{x}{b} >
    \frac{2^{k+1}b^{k+2}}{b} =
    2^{k+1}b^{k+1}
  \]

As a result, $2^{k+1}b^{k+1} < \frac{x}{b} \leq 2^{k+1}b^{k+2}$, and by the result we just proved, $G(\frac{x}{n})$ can be won by Player I during Round $k+1$. Let $s$ be the function representing this winning strategy. The game $G(\frac{x}{n})$ proceeds as follows:

\begin{center}\begin{tabular}{c|c|c}
$G(\frac{x}{n})$ & Player I & Player II \\\hline
Round $0$ & $s()$ & $n_0$ \\\hline
Round $1$ & $s(n_0)$ & $n_1$ \\\hline
$\vdots$ & $\vdots$ & $\vdots$ \\\hline
Round $k+1$ & $s(n_0,\dots,n_k)$ & $n_{k+1}$
\end{tabular}\end{center}

Since Player I won this game, we know 
  \[
    s()n_0s(n_0)\dots n_{k-1}s(n_0,\dots,n_k) < \frac{x}{n}
  \]
but 
  \[
    s()n_0s(n_0)\dots n_{k-1}s(n_0,\dots,n_k)n_{k+1} \geq \frac{x}{n}
  \]

We claim the following strategy for Player II guarantees a victory in $G(x)$: Player I opens with some arbitrary play $2\leq m \leq b$, and then uses Player II's strategy $s$ for the rest of the game.

\begin{center}\begin{tabular}{c|c|c}
$G(x)$ & Player I & Player II \\\hline
Round $0$ & $n$ & $s()$ \\\hline
Round $1$ & $n_0$ & $s(n_0)$ \\\hline
$\vdots$ & $\vdots$ & $\vdots$ \\\hline
Round $k$ & $n_{k-1}$ & $s(n_0,\dots,n_{k-1})$ \\\hline
Round $k+1$ & $n_k$ & $s(n_0,\dots,n_k)$
\end{tabular}\end{center}

Player II has now won, as
  \[
    ns()n_0s(n_0)\dots n_{k-1}s(n_0,\dots,n_{k-1})n_k < x
  \]
and 
  \[
    ns()n_0s(n_0)\dots n_{k-1}s(n_0,\dots,n_{k-1})n_ks(n_0,\dots,n_k) \geq x
  \]

This concludes the proof.
\end{proof}


\end{document}











